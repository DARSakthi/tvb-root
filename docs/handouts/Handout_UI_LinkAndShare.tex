%!TEX TS-program = pdflatex                                          %
%!TEX encoding = UTF8                                                %
%!TEX spellcheck = en-US                                             %
%
%%%%%%%%%%%%%%%%%%%%%%%%%%%%%%%%%%%%%%%%%%%%%%%%%%%%%%%%%%%%%%%%%%%%%%
% Handout_ModelingStructuralLesions.tex
% A handout to follow during the hands-on sessions 
% 
% Authors: Paula Sanz Leon
% 
% 
%%%%%%%%%%%%%%%%%%%%%%%%%%%%%%%%%%%%%%%%%%%%%%%%%%%%%%%%%%%%%%%%%%%%%%
% based on the tufte-latex template                                  %

\documentclass{tufte-handout}

%\geometry{showframe}% for debugging purposes -- displays the margins

\usepackage{amsmath}

% Set up the images/graphics \underline{\textbf{Analysis}}
\usepackage{graphicx}
\setkeys{Gin}{width=\linewidth,totalheight=\textheight,keepaspectratio}
\graphicspath{{figures/}}

\title{The Virtual Brain: Hands on Session \#2}
\date{7th June 2014} 

% The following package makes prettier tables.  
\usepackage{booktabs}

% 
%\usepackage{caption}
%\usepackage{subcaption}

% The units package provides nice, non-stacked fractions and better spacing
% for units.
\usepackage{units}
\usepackage[svgnames]{xcolor}

% The fancyvrb package lets us customize the formatting of verbatim
% environments.  We use a slightly smaller font.
\usepackage{fancyvrb}
\fvset{fontsize=\normalsize}

% Small sections of multiple columns
\usepackage{multicol}

% For adjustwidth environment
\usepackage[strict]{changepage}

% For formal definitions
\usepackage{framed}

% Resume a list
\usepackage{enumitem}

% Provides paragraphs of dummy text
\usepackage{lipsum}

% These commands are used to pretty-print LaTeX commands
\newcommand{\doccmd}[1]{\texttt{\textbackslash#1}}% command name -- adds backslash automatically
\newcommand{\docopt}[1]{\ensuremath{\langle}\textrm{\textit{#1}}\ensuremath{\rangle}}% optional command argument
\newcommand{\docarg}[1]{\textrm{\textit{#1}}}% (required) command argument
\newenvironment{docspec}{\begin{quote}\noindent}{\end{quote}}% command specification environment
\newcommand{\docenv}[1]{\textsf{#1}}% environment name
\newcommand{\docpkg}[1]{\texttt{#1}}% package name
\newcommand{\doccls}[1]{\texttt{#1}}% document class name
\newcommand{\docclsopt}[1]{\texttt{#1}}% document class option name

% environment derived from framed.sty: see leftbar environment definition
\definecolor{formalshade}{rgb}{0.95,0.95,1}
\definecolor{simulationshade}{rgb}{0.92, 1.0, 0.95}

\newenvironment{formal}{%
  \def\FrameCommand{%
    \hspace{1pt}%
    {\color{DarkBlue}\vrule width 2pt}%
    {\color{formalshade}\vrule width 4pt}%
    \colorbox{formalshade}%
  }%
  \MakeFramed{\advance\hsize-\width\FrameRestore}%
  \noindent\hspace{-4.55pt}% disable indenting first paragraph
  \begin{adjustwidth}{}{7pt}%
  \vspace{2pt}\vspace{2pt}%
}
{%
  \vspace{2pt}\end{adjustwidth}\endMakeFramed%
}

\newenvironment{simulation}{%
  \def\FrameCommand{%
    \hspace{1pt}%
    {\color{ForestGreen}\vrule width 2pt}%
    {\color{simulationshade}\vrule width 4pt}%
    \colorbox{simulationshade}%
  }%
  \MakeFramed{\advance\hsize-\width\FrameRestore}%
  \noindent\hspace{-4.55pt}% disable indenting first paragraph
  \begin{adjustwidth}{}{7pt}%
  \vspace{2pt}\vspace{2pt}%
}
{%
  \vspace{2pt}\end{adjustwidth}\endMakeFramed%
}

\begin{document}
\maketitle % this prints the handout title, author, and date

\begin{abstract}

\noindent TVB allows for a systematic exploration and manipulation of every
underlying component of a large-scale brain network model, such as the neural
mass model governing the local dynamics or the structural connectivity
constraining the space-time structure of the network couplings.
\begin{marginfigure}%
  \includegraphics[width=\linewidth]{tvb_logo_transparent_square}
  %\caption{TVB evil logo}
  \label{fig:marginfig}
\end{marginfigure}
\end{abstract}

%\printclassoptions

%\begin{fullwidth} % uncomment this environment to get full texwidth paragraphs 
%\textsc{The Virtual Brain} is a facilitating technology. It enables
%researchers from the domains of  computational neuroscience and medicine to
%study and focus on a particular problem, and directly build a model of the
%brain that can be tested under different scenarios. So, every time we require
%to perform a simulation for our work, we do not require to develop the
%underlying computational model.   
%\end{fullwidth}

\section{Objectives}\label{sec:objectives}

\newthought{This short tutorial} provides the basic steps you need to know in order to
share data with other users and/or link data to other projects.


\subsection{What's inside Project Session\_II\_ShareAndLink\_a}\label{sec:project_data}
% let's start a new thought -- a new section

\subsection{Steps: How to upload a connectivity in a zip file and share it}\label{sec:steps}


\begin{formal}
\begin{enumerate}
\item Create a project (e.g. \textit{Session\_II\_ShareAndLink\_a}). 
\item Upload a connectivity in a zip file.
\item To do that you need to go to \textsc{projects} $\rightarrow$ \textsc{data-structure}.
\item Select the connectivity you want to share. The metadata overlay will appear.
\item Go to the tab \underline{Links}.
\end{enumerate}
\end{formal} 

(Snaphsot here)

You'll see a list with all your projects. 

\begin{formal}
\begin{enumerate}[resume]
\setcounter{enumi}{5}
\item Link this datatype (connectivity) with the project you'll share (e.g. \textit{Session\_II\_ShareAndLink\_b}).
\end{enumerate}
\end{formal} 

(snaphsot here ) 

\begin{formal}
  \begin{enumerate}[resume]
    \setcounter{enumi}{6}
    \item Go to \textsc{PROJECT} $\rightarrow$ \textsc{LIST OF ALL PROJECTS}
    \item Switch to the \textit{Session\_II\_ShareAndLink\_b}.
  \end{enumerate}
\end{formal} 

Then from \textsc{PROJECT} $\rightarrow$ \textsc{basic properties} we could
share this project with another user if TVB is usde in a multiuser
configuration.  You should be able to see the connectivity matrices you linked
from project \textit{Session\_II\_ShareAndLink\_a}.


\section{More Documentation}\label{sec:more-doc}

For more documentation on The Virtual Brain, please see the following articles
\cite{Sanz-Leon_2013, Spiegler_2013, Woodman_2014, Jirsa_2010b}

\section{Support}\label{sec:support}

The official TVB webiste is \url{www.thevirtualbrain.org}.  
All the documentation and tutorials are hosted on \url{the-virtual-brain.github.io}.
You'll find our public \smallcaps{git} repository at \url{https://github.com/the-virtual-brain}. 
For questions and bug reports we have a users group \url{https://groups.google.com/forum/#!forum/tvb-users}

\bibliography{tvb_references}
\bibliographystyle{plainnat}

\end{document}


































































