%!TEX TS-program = pdflatex                                          %
%!TEX encoding = UTF8                                                %
%!TEX spellcheck = en-US                                             %
%
%%%%%%%%%%%%%%%%%%%%%%%%%%%%%%%%%%%%%%%%%%%%%%%%%%%%%%%%%%%%%%%%%%%%%%
% Handout_ModelingStructuralLesions.tex
% A handout to follow during the hands-on sessions 
% 
% Authors: Timothée Proix
% 
% 
%%%%%%%%%%%%%%%%%%%%%%%%%%%%%%%%%%%%%%%%%%%%%%%%%%%%%%%%%%%%%%%%%%%%%%
% based on the tufte-latex template                                  %

\documentclass{tufte-handout}

%\geometry{showframe}% for debugging purposes -- displays the margins

\usepackage{amsmath}

% Set up the images/graphics package
\usepackage{graphicx}
\setkeys{Gin}{width=\linewidth,totalheight=\textheight,keepaspectratio}
\graphicspath{{figures/}}

\title{The Virtual Brain: Hands on Session \#4 \\
Modeling Epilepsy using TVB}
\date{7th June 2014} 

% The following package makes prettier tables.  
\usepackage{booktabs}

% The units package provides nice, non-stacked fractions and better spacing
% for units.
\usepackage{units}

% The fancyvrb package lets us customize the formatting of verbatim
% environments.  We use a slightly smaller font.
\usepackage{fancyvrb}
\fvset{fontsize=\normalsize}

% Small sections of multiple columns
\usepackage{multicol}

% Provides paragraphs of dummy text
\usepackage{lipsum}

% Provides nice colors 
\usepackage[svgnames]{xcolor}

% These commands are used to pretty-print LaTeX commands
\newcommand{\doccmd}[1]{\texttt{\textbackslash#1}}% command name -- adds backslash automatically
\newcommand{\docopt}[1]{\ensuremath{\langle}\textrm{\textit{#1}}\ensuremath{\rangle}}% optional command argument
\newcommand{\docarg}[1]{\textrm{\textit{#1}}}% (required) command argument
\newenvironment{docspec}{\begin{quote}\noindent}{\end{quote}}% command specification environment
\newcommand{\docenv}[1]{\textsf{#1}}% environment name
\newcommand{\docpkg}[1]{\texttt{#1}}% package name
\newcommand{\doccls}[1]{\texttt{#1}}% document class name
\newcommand{\docclsopt}[1]{\texttt{#1}}% document class option name

%%%%%%%%%%%%%%%%%%%%%%%%%%%%%%%%%%%%%%%%%%%%%%%%%%%%%%%%%%%%%%%%%%%%%%%%%%%%%%
%                      The document starts here                              %
%%%%%%%%%%%%%%%%%%%%%%%%%%%%%%%%%%%%%%%%%%%%%%%%%%%%%%%%%%%%%%%%%%%%%%%%%%%%%%
\begin{document}
\maketitle % this prints the handout title, author, and date

\begin{abstract}
\noindent A small paragraph that gives context to the session, that is, 
why we are modeling this particular case.\\
The aim is to model specific patients whos data we get from neuroimaging methods and to reproduce one or several modalities . then we can try 
1) to ifind epileptogenic regions (concorance with data) 
2) to model chirugy cases and impact on the conectomr
\end{abstract}

%\printclassoptions

%\begin{fullwidth} % uncomment this environment to get full texwidth paragraphs 
 
%\end{fullwidth}

\section{Objectives}\label{sec:objectives}
from clinical point of view, see summary
The main goal of this session is to provide a clear understanding of how wen can put apatient i tvb, model the different modalities and perform soe anaylis on the results and/or export the results

Furthemore, we probably need to provide one or two references about relevant
empirical studies ...


\subsection{What's inside Project X }\label{sec:project_data}

The table should list the data inside a project. 
We provide data for simulation taken from the Human Connectom Project. Data were handled with the pipeline described earlier today.

\begin{margintable}
  \centering
  \fontfamily{ppl}\selectfont
  \begin{tabular}{ll}
    \toprule
    Datatype & Sumary info                       \\
    \midrule
    Connectivity         & \unit[X]{nodes}                    \\
    Surface              & \unit[15000 x 3]{vertices}     \\
    RegionMapping        & \unit[15000]{nodes}            \\
    EEGProjectionMatrix     & \unit[M x N]{sources x sensors} \\
    EEGSensors              & \unit[N x 4]{sensors x coordinates} \\
    MEGProjectionMatrix     & \unit[M x N]{sources x sensors} \\
    MEGSensors              & \unit[N x 4]{sensors x coordinates}\\
    IntracranialSensors     & \unit[N x 4]{sensors x coordinates} \\
    TimesSeries          & \unit[5000]{ms} \\
    \bottomrule
  \end{tabular}
  \caption{Here are the datatypes included in Project X}
  \label{tab:margintab}
\end{margintable}


% let's start a new thought -- a new section

\newthought{In this session}, we'll only go through the necessary steps
required to reproduce the data described in Table~\ref{tab:margintab}.
Describe some particularities of the project. 


\subsection{Steps: importing the data}\label{sec:import}

In your browser go to the tab that says \textsc{project}.
Figure~\ref{fig:fig} shows a snapshots of this working area.
At first step, we will import all the relevant modaliites.

A little word on the pipeline here.

Show how to import data

As a first step, you we will explore {\color{RoyalBlue}\textsc{A}}. Then, we
will do {\color{RoyalBlue} \textsc{B}} to achieve
{\color{RoyalBlue} \textsc{C}}.

\begin{enumerate}
\item Do this,
\item and that,
\item Did you get that? Then proceed with the next step
\item ...
\end{enumerate}

\subsection{Steps: exploring the epileptor model}\label{sec:epileptor}

we will look at the phase space of the epileptor model, understanding a bit what is the underlying dynamics

\subsection{Steps: Preparing a region parameter exploration }

we test several hypothesis on the position of the epileptogenic zone

\subsection{Steps: Simulating MEG spikes}

Here we simulate MEG spikes

\subsection{Steps: Surface Simulation}
Here we do a surface simulation with EEG.

\section{Results}\label{sec:results}

We look at the results of the previous sections, and we export them

\section{More Documentation}\label{sec:more-doc}
For more documentation on The Virtual Brain, please see the following articles \cite{Sanz-Leon_2013, Spiegler_2013, Jirsa_2010b}


\section{Support}\label{sec:support}

The official TVB webiste is \url{www.thevirtualbrain.org}.  
All the documentation and tutorials are hosted on \url{the-virtual-brain.github.io}.
You'll find our public \smallcaps{git} repository at \url{https://github.com/the-virtual-brain}. For questions and bug reports we have a users group \url{https://groups.google.com/forum/#!forum/tvb-users}

\bibliography{tvb_references}
\bibliographystyle{plainnat}

\end{document}