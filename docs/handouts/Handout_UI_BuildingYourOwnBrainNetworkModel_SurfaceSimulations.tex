%!TEX TS-program = pdflatex                                          %
%!TEX encoding = UTF8                                                %
%!TEX spellcheck = en-US                                             %
%
%%%%%%%%%%%%%%%%%%%%%%%%%%%%%%%%%%%%%%%%%%%%%%%%%%%%%%%%%%%%%%%%%%%%%%
% Handout_ModelingStructuralLesions.tex
% A handout to follow during the hands-on sessions 
% 
% Authors: Paula Sanz Leon
% 
% 
%%%%%%%%%%%%%%%%%%%%%%%%%%%%%%%%%%%%%%%%%%%%%%%%%%%%%%%%%%%%%%%%%%%%%%
% based on the tufte-latex template                                  %

\documentclass{tufte-handout}

%\geometry{showframe}% for debugging purposes -- displays the margins

\usepackage{amsmath}

% Set up the images/graphics \underline{\textbf{Analysis}}
\usepackage{graphicx}
\setkeys{Gin}{width=\linewidth,totalheight=\textheight,keepaspectratio}
\graphicspath{{figures/}}

\title{The Virtual Brain: Hands on Session \#1-b}
\date{7th June 2014} 

% The following package makes prettier tables.  
\usepackage{booktabs}

% 
%\usepackage{caption}
%\usepackage{subcaption}

% The units package provides nice, non-stacked fractions and better spacing
% for units.
\usepackage{units}
\usepackage[svgnames]{xcolor}

% The fancyvrb package lets us customize the formatting of verbatim
% environments.  We use a slightly smaller font.
\usepackage{fancyvrb}
\fvset{fontsize=\normalsize}

% Small sections of multiple columns
\usepackage{multicol}

% For adjustwidth environment
\usepackage[strict]{changepage}

% For formal definitions
\usepackage{framed}

% Resume a list
\usepackage{enumitem}

% Provides paragraphs of dummy text
\usepackage{lipsum}

% These commands are used to pretty-print LaTeX commands
\newcommand{\doccmd}[1]{\texttt{\textbackslash#1}}% command name -- adds backslash automatically
\newcommand{\docopt}[1]{\ensuremath{\langle}\textrm{\textit{#1}}\ensuremath{\rangle}}% optional command argument
\newcommand{\docarg}[1]{\textrm{\textit{#1}}}% (required) command argument
\newenvironment{docspec}{\begin{quote}\noindent}{\end{quote}}% command specification environment
\newcommand{\docenv}[1]{\textsf{#1}}% environment name
\newcommand{\docpkg}[1]{\texttt{#1}}% package name
\newcommand{\doccls}[1]{\texttt{#1}}% document class name
\newcommand{\docclsopt}[1]{\texttt{#1}}% document class option name

% environment derived from framed.sty: see leftbar environment definition
\definecolor{formalshade}{rgb}{0.95,0.95,1}
\definecolor{simulationshade}{rgb}{0.92, 1.0, 0.95}

\newenvironment{formal}{%
  \def\FrameCommand{%
    \hspace{1pt}%
    {\color{DarkBlue}\vrule width 2pt}%
    {\color{formalshade}\vrule width 4pt}%
    \colorbox{formalshade}%
  }%
  \MakeFramed{\advance\hsize-\width\FrameRestore}%
  \noindent\hspace{-4.55pt}% disable indenting first paragraph
  \begin{adjustwidth}{}{7pt}%
  \vspace{2pt}\vspace{2pt}%
}
{%
  \vspace{2pt}\end{adjustwidth}\endMakeFramed%
}

\newenvironment{simulation}{%
  \def\FrameCommand{%
    \hspace{1pt}%
    {\color{ForestGreen}\vrule width 2pt}%
    {\color{simulationshade}\vrule width 4pt}%
    \colorbox{simulationshade}%
  }%
  \MakeFramed{\advance\hsize-\width\FrameRestore}%
  \noindent\hspace{-4.55pt}% disable indenting first paragraph
  \begin{adjustwidth}{}{7pt}%
  \vspace{2pt}\vspace{2pt}%
}
{%
  \vspace{2pt}\end{adjustwidth}\endMakeFramed%
}

\begin{document}
\maketitle % this prints the handout title, author, and date

\begin{abstract}

\noindent TVB allows for a systematic exploration and manipulation of every
underlying component of a large-scale brain network model, such as the neural
mass model governing the local dynamics or the structural connectivity
constraining the space-time structure of the network couplings.
\begin{marginfigure}%
  \includegraphics[width=\linewidth]{tvb_logo_transparent_square}
  %\caption{TVB evil logo}
  \label{fig:marginfig}
\end{marginfigure}
\end{abstract}

%\printclassoptions

%\begin{fullwidth} % uncomment this environment to get full texwidth paragraphs 
%\textsc{The Virtual Brain} is a facilitating technology. It enables
%researchers from the domains of  computational neuroscience and medicine to
%study and focus on a particular problem, and directly build a model of the
%brain that can be tested under different scenarios. So, every time we require
%to perform a simulation for our work, we do not require to develop the
%underlying computational model.   
%\end{fullwidth}

\section{Objectives}\label{sec:objectives}

This extends the basic region simulation, covered in the previous handout, to include the folded cortical surface to the anatomical structure on which the simulation is based, if you haven't already looked at that tutorial you probably should do that now as here we only really discuss in detail the extra things that are specific to a simulation on the cortical surface.


\subsection{What's inside Project Session\_I\_BuildingYourOwnBrainNetworkModel}\label{sec:project_data}

\begin{margintable}
  \centering
  \fontfamily{ppl}\selectfont
  \begin{tabular}{ll}
    \toprule
    Datatype & Sumary info                       \\
    \midrule
    Default Connectivity & 74 {nodes}            \\
    Region Time Series   & -                     \\
    Surface Time Series  & -                     \\ 
    EEG Time-Series      &                       \\
    Local Connectivity   & -                     \\
    Fourier Spectra      & -                     \\ 
    Wavelet Spectrograms & -                     \\
    \bottomrule
  \end{tabular}
  \caption{Some of the dataypes}
  \label{tab:normaltab}
\end{margintable}


% let's start a new thought -- a new section
\newthought{In this session}, in addition to the componets discussed for a
region simulation here we introduce one new major component, that is:
underline{Cortex}, the primary component of which is a mesh surface defining
a 2d representaion of the convoluted cortical surface embedded in 3d space.
This object includes a range of ancillary information and methods required
for using it in simulations. We will also introduce two new
underline{Monitors} that make more sense in the context of surface
simulations.


\subsection{Steps: Building a Brain Network Model on a discretized spatial support.}\label{sec:steps}

The first significant thing of note about surface simulations is that certain \underline{Monitors} make a lot more sense in this context than they do at the region level, and so we'll introduce a couple new \underline{Monitors} here.

The first of these new \underline{Monitors} is called
\underline{SpatialAverage}, this \underline{Monitor} will average over the
space (nodes) of the simulation. In the case of region level simulations we
already have a situation of a relatively small number of nodes, with each one
representing a fairly large chunk of brain. In surface simulations on the
other hand we can easily have tens of thousands of nodes, and reducing this by
averaging over sensible collections of these nodes can be valuable. The basic
mechanism is general, in the sense that the nodes can be broken up into any
non-overlapping, complete, set of sets -- said another way, each node can only
be counted in one collection and all nodes must be in one collection. As a
concrete example, even in surface simulations information regarding a break up
into regions exists, and this breakup, where each cortical mesh node belongs
to one and only one region can be used to define the spatial average. In fact
this is the default behaviour of the \underline{SpatialAverage} monitor when
applied to a surface simulation, that is it averages over nodes and returns
region based time-series. An alternative is to spatially average the signals
according to the hemisphers the nodes belong two.  When applied to region
simulation distinct spatial masks, or index vectors, can be used such as
\underline{hemispheres} which, as you may guess, will average the left nodes
and right nodes separetly resulting in two time-series; and,
\underline{cortical}, which will average all the cortical nodes and those
repsresenting subcortical structuresin two different time series.


The second of these new \underline{Monitors}, which is an instantiation of a
biophysical measurement process, is called EEG. This monitor, hopefully also
unsurprisingly, returns the EEG signal resulting from the simulated neural
dynamics. EEG signals measured on the scalp depend strongly on the location
and orientation of the underlying neural sources, which is why this monitor is
more realistic and useful in the case of surface based simulations -- where
the simulation is run on the explicit geometry of the cortex, which can
potentially have been obtained from a specific individual's brain. In addition
a simulation being built on the specific anatomical structure of an individual
subject, the specific electrodes used in experimental work can also be
incorporated, providing as direct as possible a link between simulation and
experiment. Here, we'll once again rely on TVB's defaults, where the default
\underline{SpatialAverage} monitor will return region level time-series and
the EEG monitor will return a relatively standard 62 channel set based on the
10-20 system.

With the cortical surface as spatial support, each vertex represents a neural
population and a local connectivity kernel describes the exponential decay in
the probability of connectivity – typically spanning a few millimetres from a
given focal point.

A (homogeneous) connectivity kernel can be used to capture how populations a
certain distance apart affect one another. The kernel function can be strictly
positive (eg, a Gaussian or Laplacian kernel). Here, we'll use TVB's default
local connectivity, which is a Gaussian kernel spanning 40 mm. Alternatively,
it can have both positive and negative components (distal inhibitory effects)
like a double Gaussian distribution, also known as the Mexican hat function.
We'll see how to define our own local connectivity kernel later on on this
sesssion.

\begin{simulation}
\begin{itemize}
 \item The resulting simulation is \textit{AnatomyOfASurfaceSimulation\_a} and  \textit{AnatomyOfASurfaceSimulation\_b} have the same parameters as \textit{AnatomyOfARegionSimulation\_a}; and \textit{AnatomyOfASurfaceSimulation\_c} and \textit{AnatomyOfASurfaceSimulation\_d} have the same parameters as \textit{AnatomyOfARegionSimulation\_b} however the \underline{local coupling strength} takes different values. The effect of this parameter can be seen in the wavelet spectrogram.
\end{itemize}
\end{simulation}

\subsection{Steps: How to define your own local connectivity kernel.}\label{sec:steps}


We first should go to \textsc{Connectivity} $\rightarrow$ {Local Connectivity}. 
In this area we'll build to different kernels: a Gaussian and a Mexican Hat kernel. 

\begin{formal}
\begin{enumerate}
\item Select the equation defining the spatial profile of your connectivity. Here, we'll only change the standard deviation of the kernel, which changes its width. If you want you can try changing other parameters, see their effects and save this new connectivity.
\item Select a cutoff distance, that is, the distance up to which a given node is connected to its neighbourhood.
\end{enumerate}
\end{formal}

We have on the right a representation of the kernel for a quick idea of how
the Local Connectivity we've just specified will be represented on the
surface. This plots the local connectivity function with different sampling
based on the distribution of edge lengths in your mesh surface. If all the
lines don't, at least mostly, overlap then you've probably specified a
function with structure that is too fine for the resolution of your mesh
surface. Also, you want the function to have essentially dropped to zero by
the cutoff distance.

(Snapshot of gaussian kernel around here)

\begin{formal}
\begin{enumerate}
\item Select the equation Merxican Hat equation. Here we changed all the default parameters, incuding the cuttoff distance: this kernel is very narrow and has a negative part that will play the role of inhibition.
\end{enumerate}
\end{formal}

So, the regularized mesh can support, in principle, arbitrary forms for the
local connectivity kernel. Coupled across the realistic surface geometry this
allows for a detailed investigation of the local connectivity’s effects on
larger scale dynamics modelled by neural fields.



\section{More Documentation}\label{sec:more-doc}
For more documentation on The Virtual Brain, please see the following articles \cite{Sanz-Leon_2013, Spiegler_2013, Woodman_2014, Jirsa_2010b}


\section{Support}\label{sec:support}

The official TVB webiste is \url{www.thevirtualbrain.org}.  
All the documentation and tutorials are hosted on \url{the-virtual-brain.github.io}.
You'll find our public \smallcaps{git} repository at \url{https://github.com/the-virtual-brain}. 
For questions and bug reports we have a users group \url{https://groups.google.com/forum/#!forum/tvb-users}

\bibliography{tvb_references}
\bibliographystyle{plainnat}

\end{document}

















